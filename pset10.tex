\documentclass[a4paper]{exam}

\usepackage{amsfonts,amsmath,amssymb,amsthm}
\usepackage{geometry}

\usepackage{draftwatermark}
\SetWatermarkText{Sample Solution}
\SetWatermarkScale{3}
\SetWatermarkLightness{.95}
\printanswers

\newcommand\mbb[1]{\ensuremath{\mathbb{#1}}}
\newcommand\mcal[1]{\ensuremath{\mathcal{#1}}}
\newcommand\Z{\ensuremath{\mbb{Z}}}
\newcommand\R{\ensuremath{\mbb{R}}}
\newcommand\Q{\ensuremath{\mbb{Q}}}

\title{Problem Set 10: Cardinality Of Sets}
\author{CS/MATH 113 Discrete Mathematics}
\date{Spring 2024}

\boxedpoints

\printanswers

\begin{document}
\maketitle

\begin{questions}
\question 
  Show that the two given sets have equal cardinality by describing a bijection from one to the other. Describe your bijection with a formula (not as a table). You need not include a proof of bijection.
  \begin{parts}
  \part The set of even integers and the set of odd integers
    \begin{solution}
      $f:\mbb{E}\to\mbb{O},\quad f(x) = x + 1$
    \end{solution}
    
  \part $\Z$ and $S = \{..., \frac{1}{8}, \frac{1}{4}, \frac{1}{2}, 1, 2, 4, 8, 16, ...\}$
    \begin{solution}
      $f:\mbb{Z}\to S,\quad f(x) = 2^x$
    \end{solution}

  \part $\Z^+$ and $S = \{2^n : n \in \Z^+\}$
    \begin{solution}
      $f:\mbb{Z}^+\to S,\quad f(n) = 2^n$
    \end{solution}
    
  \part $A = \{3k : k \in \Z\}$ and $B = \{7k : k \in \Z\}$
    \begin{solution}
      $f:A\to B,\quad f(x) = \frac{7}{3}x$
    \end{solution}
    
  \part $\R$ and $S=\{x\in\R\mid x>0 \}$
    \begin{solution}
      $f:\mbb{R}\to S,\quad f(x) = e^x$
    \end{solution}
    
  \part $\R$ and $S=\{x\in\R\mid x>\sqrt{2} \}$
    \begin{solution}
      $f:\mbb{R}\to S,\quad f(x) = e^x+\sqrt{2}$
    \end{solution}
  \end{parts}

  \newpage
\question Prove or disprove each of the following statements.
  \begin{parts}
  \part If $A = \{X \subseteq \Z^+ \mid X \text{ is finite}\}$, then $|A| = \aleph_0$.
    \begin{solution}
      We provide a direct proof of the statement, based on enumerating subsets of a given cardinality. The argument uses the one used to prove that $\mbb{Q}$ is countable.

      \begin{proof} $A$ is countable.
        
        Observe that all the subsets of $\Z^+$ with a given cardinality, $n$, can be listed in some order, i.e., as a sequence.\\
        That is, there are countably many subsets of $\Z^+$ of a given cardinality.\\
        Consider a matrix in which the $i$-th row, $i\ge 0$, contains all the subsets of $\Z^+$ whose cardinality is $i$.\\
        This matrix contains all the elements of $A$.\\
        $A$ is countable using the same argument as the one used to prove that \Q\ is countable.
      \end{proof}
    \end{solution}
    
  \part The set $A = \{(m,n) \in \Z^+ \times \Z^+ \mid m \leq n\}$ is countably infinite.

    \begin{solution} 
    We provide a direct proof of the statement, making use of the result that a subset of a countable set is countable.
    \begin{proof} $A$ is countably infinite.
      
      Observe that $A\subseteq \Z^+ \times \Z^+$.\\
      As shown in the proof that \Q\ is countable, $\Z^+ \times \Z^+$ is countable.\\
      Therefore, $A$ is countable.\\
      As $m$ and $n$ in the description of $A$ are unbounded, $A$ is infinite.\\
      Therefore, $A$ is countably infinite.
    \end{proof}
    % We can establish a bijection between $A$ and $\Z^+$, which proves that $A$ is countably infinite.

      % Consider the function $f: A \rightarrow \Z^+$ defined as follows:
      % \[ f(m,n) = \frac{1}{2}(m + n - 2)(m + n - 1) + m \]

      % To show that this function is
      %   one-to-one and onto, we merely need to show that the range
      %   of values for x + 1 picks up precisely where the range of
      %   values for x left off, i.e., that $f(x-1, 1)+1 = f(1, x)$. Not only would this be one-to-one, it would be onto as well. We have
      %  \[f(x - 1, 1) + 1\]
      %  \[= \frac{(x-2)(x-1)}{2}+(x-1)+1\]
      %   \[= \frac{x^2 - x+2}{2}\]
      %   \[= \frac{(x-1)x}{2}+1\]
      %   \[= f(1, x)\]
      %   Hence, this is a bijection between A and $\Z^+$, meaning A is countably infinite.
    \end{solution}
    
  \part The set $\Z \times \Q$ is countably infinite.
    \begin{solution}
      We provide a direct proof, using the result that the product of countable sets is also countable, as illustrated in the proof that \Q\ is countable.

      \begin{proof}$\Z \times \Q$ is countably infinite.
        
        \Z\ and \Q\ are countably infinite.\\
        Therefore $\Z \times \Q$ is countably infinite.
      \end{proof}
    \end{solution}
  
      \part If $A \subseteq B$ and there is an injection $g : B \to A$, then $|A| = |B|$.
        \begin{solution}
          We provide a direct proof, first showing that $|B|\le |A|$, then that $|A|\le |B|$. These establish that $|A|=|B|$.
          \begin{proof} The given statement is true.

            We know that an injection exists from $B$ to $A$.\\
            Using definition 2 from Section 2.5 of the textbook, $|B|\le |A|$.

            Because $A \subseteq B$, there exists an injection $f : A \to B$, namely $f(a)=a$.\\
            Then, $|A|\le |B|$.
    
            Combining both inequalities, we have $|A| = |B|$.
          \end{proof}
        \end{solution}
    
    \part If $|A| = |B|$ and $|B| = |C|$, then $|A| = |C|$.
    \begin{solution}
          We provide a direct proof, using the results that two sets have the same cardinality iff a bijection exists between them, and that the composition of two bijections is a bijection.
          \begin{proof} $(|A| = |B| \land |B| = |C|) \implies |A| = |C|$
            
            Because $|A|=|B|$, there exists a bijection, $f : A \to B$.\\
            Similarly, there exists a bijection $g : B\to C$.\\
            These can be composed to yield, $g \circ f:A\to C$, which is a bijection.\\
            Thus, $|A| = |C|$.
          \end{proof}
    \end{solution}
    
    \part If $A$ and $B$ are sets with $|A| = |B|$, then $|\mathcal{P}(A)| = |\mathcal{P}(B)|$.
    \begin{solution}
    Let $|A| = |B| = n$. Consider a bijection $f: A \to B$. Then, for each subset $X \subseteq A$, the set $f(X) = \{f(x) : x \in X\}$ is a subset of $B$. 
    
    Define a function $g: \mathcal{P}(A) \to \mathcal{P}(B)$ as follows:
    \[ g(X) = f(X) \]
    
    Clearly, $g$ is a bijection, since $f$ is a bijection and $g$ simply maps each subset of $A$ to its corresponding subset of $B$. Therefore, $|\mathcal{P}(A)| = |\mathcal{P}(B)|$.
    \end{solution}
\end{parts}

  
\end{questions}
\end{document}
%%% Local Variables:
%%% mode: latex
%%% TeX-master: t
%%% End: