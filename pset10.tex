\documentclass[a4paper]{exam}

\usepackage{geometry}
\usepackage{amsmath}
\usepackage{amsfonts}

\usepackage{draftwatermark}
\SetWatermarkText{Sample Solution}
\SetWatermarkScale{3}
\SetWatermarkLightness{.95}
\printanswers

\newcommand\Z{\ensuremath{\mathbb{Z}}}
\newcommand\R{\ensuremath{\mathbb{R}}}
\newcommand\Q{\ensuremath{\mathbb{Q}}}

\title{Problem Set 10: Cardinality Of Sets}
\author{CS/MATH 113 Discrete Mathematics}
\date{Spring 2024}

\boxedpoints

\printanswers

\begin{document}
\maketitle

\begin{questions}
\question 
  Show that the two given sets have equal cardinality by describing a bijection from one to the other. Describe your bijection with a formula (not as a table). You need not include a proof of bijection.
  \begin{parts}
  \part The set of even integers and the set of odd integers
    \begin{solution}
      $f(x) = x + 1$
    \end{solution}
    
  \part $\Z$ and $S = \{..., \frac{1}{8}, \frac{1}{4}, \frac{1}{2}, 1, 2, 4, 8, 16, ...\}$
    \begin{solution}
      $f(x) = 2^x$
    \end{solution}

  \part $\Z^+$ and $S = \{2^n : n \in \Z^+\}$
    \begin{solution}
      $f(n) = 2^n$
    \end{solution}
    
  \part $A = \{3k : k \in \Z\}$ and $B = \{7k : k \in \Z\}$
    \begin{solution}
      $f(x) = \frac{7}{3}x$
    \end{solution}
    
  \part $\R$ and $S=\{x\in\R\mid x>0 \}$
    \begin{solution}
      $f(x) = e^x$
    \end{solution}
    
  \part $\R$ and $S=\{x\in\R\mid x>\sqrt{2} \}$
    \begin{solution}
      $f(x) = e^x+\sqrt{2}$
    \end{solution}
  \end{parts}

\question Prove or disprove each of the following statements.
  \begin{parts}
  \part If $A = \{X \subseteq \Z^+ \mid X \text{ is finite}\}$, then $|A| = \aleph_0$.
    \begin{solution}
      We can prove this statement by showing a bijection between $A$ and $\Z^+$, implying that they have the same cardinality.

      Consider the function $f: A \rightarrow \Z^+$ defined as follows:
      \[ f(X) = |X| \]

      Where $|X|$ denotes the cardinality of the set $X$. This function maps each finite subset of $\Z^+$ to its cardinality, which is a positive integer. 

      This function is injective because distinct finite sets have distinct cardinalities. It is also surjective because every positive integer $n$ can be the cardinality of some finite subset of $\Z^+$ (e.g., the set $\{1, 2, \ldots, n\}$).

      Therefore, $f$ is a bijection between $A$ and $\Z^+$, implying that $|A| = |\Z^+| = \aleph_0$.
    \end{solution}
  \part The set $A = \{(m,n) \in \Z^+ \times \Z^+ \mid m \leq n\}$ is countably infinite.
    \begin{solution}
      We can establish a bijection between $A$ and $\Z^+$, which proves that $A$ is countably infinite.

      Consider the function $f: A \rightarrow \Z^+$ defined as follows:
      \[ f(m,n) = \frac{1}{2}(m + n - 2)(m + n - 1) + m \]

      To show that this function is
        one-to-one and onto, we merely need to show that the range
        of values for x + 1 picks up precisely where the range of
        values for x left off, i.e., that $f(x-1, 1)+1 = f(1, x)$. Not only would this be one-to-one, it would be onto as well. We have
       \[f(x - 1, 1) + 1\]
       \[= \frac{(x-2)(x-1)}{2}+(x-1)+1\]
        \[= \frac{x^2 - x+2}{2}\]
        \[= \frac{(x-1)x}{2}+1\]
        \[= f(1, x)\]
        Hence, this is a bijection between A and $\Z^+$, meaning A is countably infinite.
    \end{solution}
    
  \part The set $\Z \times \Q$ is countably infinite.
    \begin{solution}
      We know that both $\Z$ and $\Q$ are countably infinite, and we know that the Cartesian
product of two countably infinite sets is again countably infinite. Therefore $\Z \times \Q$ is
countably infinite.
    \end{solution}
  
      \part If $A \subseteq B$ and there is an injection $g : B \to A$, then $|A| = |B|$.
    \begin{solution}
    Suppose there is an injection $g : B \to A$. Then, for each $a \in A$, there exists at most one $b \in B$ such that $g(b) = a$. Since $g$ is an injection, each element $a \in A$ is mapped to by at most one element $b \in B$. Therefore, $|A| \leq |B|$.
    
    On the other hand, since $A \subseteq B$, there exists an injection $f : A \to B$ defined by the inclusion map, where $f(a) = a$ for all $a \in A$. Since $f$ is an injection, each element $a \in A$ is mapped to a distinct element $b = f(a) \in B$. Therefore, $|B| \leq |A|$.
    
    Combining both inequalities, we have $|A| = |B|$.
    \end{solution}
    
    \part If $|A| = |B|$ and $|B| = |C|$, then $|A| = |C|$.
    \begin{solution}
    By definition, we have bijections $f : A \to B$ and $g : B\to C$ . Similarly, $g \circ f$ is a valid bijection
    from A to C, as it is the composition of two bijections. Thus, $|A| = |C|$.

    \end{solution}
    
    \part If $A$ and $B$ are sets with $|A| = |B|$, then $|\mathcal{P}(A)| = |\mathcal{P}(B)|$.
    \begin{solution}
    Let $|A| = |B| = n$. Consider a bijection $f: A \to B$. Then, for each subset $X \subseteq A$, the set $f(X) = \{f(x) : x \in X\}$ is a subset of $B$. 
    
    Define a function $g: \mathcal{P}(A) \to \mathcal{P}(B)$ as follows:
    \[ g(X) = f(X) \]
    
    Clearly, $g$ is a bijection, since $f$ is a bijection and $g$ simply maps each subset of $A$ to its corresponding subset of $B$. Therefore, $|\mathcal{P}(A)| = |\mathcal{P}(B)|$.
    \end{solution}
\end{parts}

  
\end{questions}
\end{document}
%%% Local Variables:
%%% mode: latex
%%% TeX-master: t
%%% End: